\section{Changes in Relative Power during Sleep}

To observe changes of each relative power in frequency band during sleep (comparing the
five sleep stages in a figure with the same max value for the five topograms which must
be indicated). You can select only two spectral methods: selecting one of them with
similar performance in most methods and another different to observe the influence of the
estimator in checking these changes (each method in separated figures). Select the Welch
periodogram and AR with order selected by MDL criteria.

\subsection*{Welch Periodogram vs AR MDL}
In this exercise we will:

\begin{itemize}
    \item Compare the power changes across all stages of sleep 
    \item Both for Welch Periodogram and AR MDL
\end{itemize}


\begin{figure}[!h]
    \centering
    \includegraphics[width=0.9\textwidth]{img/workflow}
    \caption{Workflow done to achieve the exercise.}
\end{figure}

\vspace{1cm}
\noindent \subsubsection*{With that logic, we implemented the following code.}

\begin{lstlisting}[caption={Code used for batch processing both PSD methods on different stages of sleep for different frequency bands.}]
for i = 1:num_stages
    stage_name = stages{i};
    fprintf('  Processing Stage: %s\n', stage_name);
    
    % Load Data
    file_name = sprintf('%s%s.mat', data_path, stage_name);
    data = load(file_name);
    signals = data.signals;
    
    % Temporary storage for all epochs within this stage (Channels x Epochs)
    stage_power_welch = struct();
    stage_power_ar = struct();
    for b = 1:num_bands
        stage_power_welch.(band_names{b}) = zeros(num_channels, num_epochs);
        stage_power_ar.(band_names{b}) = zeros(num_channels, num_epochs);
    end

    % Loop through Epochs 
    for j = 1:num_epochs
        
        % Extract Epoch
        start_idx = (j-1) * nsamples_epoch + 1;
        end_idx = j * nsamples_epoch;
        eeg_epoch_raw = signals(start_idx:end_idx, eeg_channel_indices);
        
        % Filter Epoch
        eeg_epoch_filtered = filtfilt(b_filter, a_filter, eeg_epoch_raw);
        
        % Method 1: Welch
        [PxxW, f_welch] = pwelch(eeg_epoch_filtered, samples_segment, overlap_samples, nfft, fs, 'onesided');
        rel_power_w = calculate_relative_powers(PxxW, f_welch, bands);
        
        % Method 2: AR (MDL)
        avg_nmdl = get_ar_mdl_order(eeg_epoch_filtered, fs, max_ar_order);
        [PxxAR, f_ar] = pburg(eeg_epoch_filtered, avg_nmdl, nfft, fs, 'onesided'); 
        rel_power_ar = calculate_relative_powers(PxxAR, f_ar, bands);
        
        % Store results for this epoch
        for b = 1:num_bands
            stage_power_welch.(band_names{b})(:, j) = rel_power_w.(band_names{b});
            stage_power_ar.(band_names{b})(:, j) = rel_power_ar.(band_names{b});
        end
    end
    
    % Average Epochs 
    for b = 1:num_bands
        results_welch.(band_names{b})(:, i) = mean(stage_power_welch.(band_names{b}), 2);
        results_ar_mdl.(band_names{b})(:, i) = mean(stage_power_ar.(band_names{b}), 2);
    end
    
end
\end{lstlisting}

Once processed all the data, we plotted the results with the following code:

\begin{lstlisting}[caption={Customized plotting function for laying topogram plots horizontally.}]
for b = 1:num_bands
    band_name = band_names{b};
    
    % Welch results
    plot_comparison_topograms(results_welch.(band_name), stages, band_name, 'Welch', ['report/img/topogram_welch_', band_name, '.png']);
    
    % AR(MDL) results
    plot_comparison_topograms(results_ar_mdl.(band_name), stages, band_name, 'AR (MDL)', ['report/img/topogram_ar_', band_name, '.png']);
end
\end{lstlisting}

\subsection{Results}
After applying the defined workflow to the 5 stages of sleep, 
we obtained the following results for the previously defined bands.


\begin{figure}[!h]
    \begin{subfigure}{0.5\textwidth}
        \centering
        \includegraphics[width=\linewidth]{img/topogram_ar_low_delta}
    \end{subfigure}
    \hfill
     \begin{subfigure}{0.5\textwidth} 
        \centering
        \includegraphics[width=\linewidth]{img/topogram_welch_low_delta}
    \end{subfigure}
\end{figure}

\begin{figure}[!h]
    \begin{subfigure}{0.5\textwidth}
        \centering
        \includegraphics[width=\linewidth]{img/topogram_ar_delta}
    \end{subfigure}
    \hfill
    \begin{subfigure}{0.5\textwidth} 
        \centering
        \includegraphics[width=\linewidth]{img/topogram_welch_delta}
    \end{subfigure}
\end{figure}

\begin{figure}[!h]
    \begin{subfigure}{0.5\textwidth}
        \centering
        \includegraphics[width=\linewidth]{img/topogram_ar_theta}
    \end{subfigure}
    \hfill
     \begin{subfigure}{0.5\textwidth} 
        \centering
        \includegraphics[width=\linewidth]{img/topogram_welch_theta}
    \end{subfigure}
\end{figure}

\begin{figure}[!h]
    \begin{subfigure}{0.5\textwidth}
        \centering
        \includegraphics[width=\linewidth]{img/topogram_ar_alpha}
    \end{subfigure}
    \hfill
     \begin{subfigure}{0.5\textwidth} 
        \centering
        \includegraphics[width=\linewidth]{img/topogram_welch_alpha}
    \end{subfigure}
\end{figure}

\begin{figure}[!h]
    \begin{subfigure}{0.5\textwidth}
        \centering
        \includegraphics[width=\linewidth]{img/topogram_ar_sigma}
    \end{subfigure}
    \hfill
     \begin{subfigure}{0.5\textwidth} 
        \centering
        \includegraphics[width=\linewidth]{img/topogram_welch_sigma}
    \end{subfigure}
\end{figure}

\begin{figure}[!h]
    \begin{subfigure}{0.5\textwidth}
        \centering
        \includegraphics[width=\linewidth]{img/topogram_ar_beta}
    \end{subfigure}
    \hfill
     \begin{subfigure}{0.5\textwidth} 
        \centering
        \includegraphics[width=\linewidth]{img/topogram_welch_beta}
    \end{subfigure}
\end{figure}


\subsection{Discussion}
The obtained result match very well with what is described in the literature.
Showing a clear correlation with Power Sprectral Density and activity in each specific 
case. 
In our opinion, the stage S1 (N1) is very interesting because appears to have oscillations 
from both stages (sleep/awake) and has activity of all the bands except from alpha band.\\

Both Welch Periodogram and AR (MDL) PSD methods perform very similarly. However, by observing 
topogram representations of PSD power/electrode, we think that the AR method is more prone to 
saturate (giving less visual evidence of differences across electrodes), whereas the Welch 
method seems more balanced.
