\section{Changes in Relative Power during Sleep}

To observe changes of each relative power in frequency band during sleep (comparing the
five sleep stages in a figure with the same max value for the five topograms which must
be indicated). You can select only two spectral methods: selecting one of them with
similar performance in most methods and another different to observe the influence of the
estimator in checking these changes (each method in separated figures). Select the Welch
periodogram and AR with order selected by MDL criteria.

\subsection*{Welch Periodogram vs AR MDL}
In this exercise we will:

\begin{itemize}
    \item Compare the power changes across all stages of sleep 
    \item Both for Welch Periodogram and AR MDL
\end{itemize}


\begin{figure}[!h]
    \centering
    \includegraphics[width=0.9\textwidth]{img/workflow}
    \caption{Workflow done to achieve the exercise.}
\end{figure}


\subsection{Defined Frequency Bands}
\label{sec:defined_bands}

\noindent For the quantitative analysis of the EEG signal across sleep stages, the Power Spectral Density (PSD) is segmented into the following 
frequency bands. The power contained within these ranges correlates directly to distinct mental or sleep states.

\vspace{0.5em} % Adds a little vertical space

\begin{itemize}
    \item \textbf{Low Delta ($0.5-2$ Hz):} A sub-band of Delta often associated with the deepest stages of non-REM sleep (N3) and slow wave activity.
    \item \textbf{Delta ($\delta$, $0.5-4$ Hz):} The hallmark of deep sleep (SWS/N3). Characterized by high amplitude and slow activity.
    \item \textbf{Theta ($\theta$, $4-8$ Hz):} Predominant activity during the transition into sleep and light sleep stages (N1 and N2).
    \item \textbf{Alpha ($\alpha$, $8-13$ Hz):} Associated with quiet, relaxed wakefulness, typically seen over the posterior scalp with eyes closed.
    \item \textbf{Sigma ($\sigma$, $11-16$ Hz):} This range encompasses the **Sleep Spindles**, which are a defining feature of Stage N2 sleep.
    \item \textbf{Beta ($\beta$, $16-35$ Hz):} High-frequency activity linked to active concentration, complex thought, and wakefulness.
\end{itemize}

\subsection{Results}
After applying the defined workflow to the 5 stages of sleep, 
we obtained the following results for the previously defined bands.


\begin{figure}[!h]
    \begin{subfigure}{0.5\textwidth}
        \centering
        \includegraphics[width=\linewidth]{img/topogram_ar_low_delta}
    \end{subfigure}
    \hfill
     \begin{subfigure}{0.5\textwidth} 
        \centering
        \includegraphics[width=\linewidth]{img/topogram_welch_low_delta}
    \end{subfigure}
\end{figure}

\begin{figure}[!h]
    \begin{subfigure}{0.5\textwidth}
        \centering
        \includegraphics[width=\linewidth]{img/topogram_ar_delta}
    \end{subfigure}
    \hfill
    \begin{subfigure}{0.5\textwidth} 
        \centering
        \includegraphics[width=\linewidth]{img/topogram_welch_delta}
    \end{subfigure}
\end{figure}

\begin{figure}[!h]
    \begin{subfigure}{0.5\textwidth}
        \centering
        \includegraphics[width=\linewidth]{img/topogram_ar_theta}
    \end{subfigure}
    \hfill
     \begin{subfigure}{0.5\textwidth} 
        \centering
        \includegraphics[width=\linewidth]{img/topogram_welch_theta}
    \end{subfigure}
\end{figure}

\begin{figure}[!h]
    \begin{subfigure}{0.5\textwidth}
        \centering
        \includegraphics[width=\linewidth]{img/topogram_ar_alpha}
    \end{subfigure}
    \hfill
     \begin{subfigure}{0.5\textwidth} 
        \centering
        \includegraphics[width=\linewidth]{img/topogram_welch_alpha}
    \end{subfigure}
\end{figure}

\begin{figure}[!h]
    \begin{subfigure}{0.5\textwidth}
        \centering
        \includegraphics[width=\linewidth]{img/topogram_ar_sigma}
    \end{subfigure}
    \hfill
     \begin{subfigure}{0.5\textwidth} 
        \centering
        \includegraphics[width=\linewidth]{img/topogram_welch_sigma}
    \end{subfigure}
\end{figure}

\begin{figure}[!h]
    \begin{subfigure}{0.5\textwidth}
        \centering
        \includegraphics[width=\linewidth]{img/topogram_ar_beta}
    \end{subfigure}
    \hfill
     \begin{subfigure}{0.5\textwidth} 
        \centering
        \includegraphics[width=\linewidth]{img/topogram_welch_beta}
    \end{subfigure}
\end{figure}


\subsection{Discussion}
The obtained result match very well with what is described in the literature.
Showing a clear correlation with Power Sprectral Density and activity in each specific 
case. 
In our opinion, the stage S1 (N1) is very interesting because appears to have oscillations 
from both stages (sleep/awake) and has activity of all the bands except from alpha band.\\

Both Welch Periodogram and AR (MDL) PSD methods perform very similarly. However, by observing 
topogram representations of PSD power/electrode, we think that the AR method is more prone to 
saturate (giving less visual evidence of differences across electrodes), whereas the Welch 
method seems more balanced.
