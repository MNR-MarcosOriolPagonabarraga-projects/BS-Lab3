\section{Feasibility Assessment}

\subsection{a) Compare the PSD functions for the channels Fz, Cz and Pz obtained in all 7 spectral methods in a figure (consider only the epoch 1).}
\subsubsection*{Epoch 1: corresponds to the first 5 seconds}
\begin{itemize}
    \item First, we compute the PSD for the last 19 EEG channels, for the first 5 seconds (epoch 1). 
    \item Then select from the third channel (this included) until the last one, in total 29 channels. This is because the first two correspond to the two ocular channels.
\end{itemize}

\begin{lstlisting}
fs = 100; % Sampling frequency (Hz)
N = 5*fs; % samples in one 5-seconds epoch

% Compute all PSD from the 19 EEG channels (ephoch 1)
% We select the first 5 seconds epoch and the last 19EEG channels:
eegi = signals(1:N,3:21); % From 1 to 500 samples (5 seconds)
\end{lstlisting}

\vspace{0.7cm}
Then, it is important to filter the signal. As frequency bands come from 0.5 Hz to 35 Hz, it is convenient to remove the very low frequency components associated 
mainly with the movement and other brain activities different to delta rhythms. By doing this it will permit to estimate better the PSD.
\vspace{0.7cm}

\begin{lstlisting}
% Apply high-pass filter
[b,a] = ellip(6,0.5,40,.4/(fs/2),'high');
eeg = filtfilt(b,a,eegi);
\end{lstlisting}

\vspace{0.7cm}
\noindent With the data filtered, we will proceed to compute the PSD using the 7 spectral methods, defined in ~\ref{sec:psd_methods}.
\vspace{0.7cm}

\begin{lstlisting}[caption={Code used for running the different Power Spectral Methods.}]
% Method 1 - Periodogram, considering a FFT with 1000 points:
[Pxx,f]=periodogram(eeg,[],1000,fs,'onesided');

% Method 2 - Periodogram using a Hamming window
[PxxH,f]=periodogram(eeg,hamming(500),1000,fs,'onesided');

% Method 3 - averaged Welch periodogram
number_segments = 4;
overlap = 0.5;
samples_segment = floor(length(eeg)/(number_segments-(number_segments-1)*overlap));
[PxxW,f]=pwelch(eeg,samples_segment,floor(samples_segment*overlap),1000,fs,'onesided'); 

% Method 4 - Burg method with low order of 2
[Parb2,f]= pburg(eeg,2,1000,fs,'onesided');

% Method 5 - Burg method with high order of 50
[Parb50,f]= pburg(eeg,50,1000,fs,'onesided');

% Method 6 and 7 - AIC and MDL, respectively
nch = size(eeg, 2);
nnaic = zeros(nch, 1);
nnmdl = zeros(nch, 1);

for n = 1:nch
 data1= iddata(eeg(1:N/2, n),[],fs);
 data2= iddata(eeg(N/2 + 1: N,n),[], fs);
 V1=arxstruc(data1,data2, (1:60));
 nnaic(n)=selstruc(V1,'aic');
 nnmdl(n)=selstruc(V1,'mdl');
end

% Method 6 (AIC)
naic = round(mean(nnaic));
[Parbaic,f] = pburg(eeg,naic,1000,fs,'onesided');

% Method 7 (MDL)
nmdl=round(mean(nnmdl));
[Parbmdl,f]= pburg(eeg,nmdl,1000,fs,'onesided');
\end{lstlisting}

We now select the channels indicated (Fz, Cz and Pz). Since the data we are working with ("eeg") has already 19 channels 
(the two first, corresponding to the ocular channels, have been removed previously, variable "eegi"), we can assign the 
following correspondences between channelsa and rows from "eegi":

\begin{itemize}
    \item Fz: 5th channel
    \item Cz: 10th channel
    \item Pz: 15th channel
\end{itemize}

\begin{figure}[H]
    \centering
    \includegraphics[width=\textwidth]{img/ex1/ps_methods}
    \caption{7 PSD methods applied to the epoch 1 of S1 data.}
\end{figure}


\subsection{b) Compare the topograms of all 7 spectral methods in a figure for each relative power (indicate the max value considered in each of both figures).}
The frequency bands and their corresponding limits are the ones indicated in ~\ref{sec:defined_bands}.
Next, we present the code used for calculating the PSD methods for the defined bands.

\begin{lstlisting}[caption={Matlab code used for running the different PSD for each frequency band.}]
% Define the 7 spectral methods and their names
PSD_methods = {Pxx, PxxH, PxxW, Parb2, Parb50, Parbaic, Parbmdl};
method_names = {'Periodogram', 'Hamming', 'Welch', 'Burg (p=2)', 'Burg (p=50)', 'Burg (AIC)', 'Burg (MDL)'};
num_methods = length(PSD_methods);
num_channels = size(eeg, 2);

% Define all 6 frequency bands and their limits
Band_Limits = {
    'Low Delta (0.5 - 2 Hz)', [0.5, 2.0];
    'Delta (2 - 4 Hz)',       [2.0, 4.0];
    'Theta (4 - 7.5 Hz)',     [4.0, 7.5];
    'Alpha (7.5 - 12 Hz)',    [7.5, 12.0];
    'Sigma (12 - 15 Hz)',     [12.0, 15.0];
    'Beta (15 - 35 Hz)',      [15.0, 35.0];
};

num_bands = size(Band_Limits, 1);

% Total frequency range for normalization
f_min_total = 0.5; % Hz
f_max_total = 30.0; % Hz 
idx_total = find(f >= f_min_total & f <= f_max_total); % f: frequency vector

for band_idx = 1:num_bands
    
    % --- Get Current Band Info ---
    band_name = Band_Limits{band_idx, 1};
    f_min_band = Band_Limits{band_idx, 2}(1);
    f_max_band = Band_Limits{band_idx, 2}(2);

    disp(['Processing band: ', band_name]);

    % Find frequency indices for the current band
    idx_band = find(f >= f_min_band & f <= f_max_band);
    
   % Initialize matrix to store relative power for all methods (Rows=Methods, Cols=Channels)
   % Relative power calculated as before
    Relative_Power_Matrix = zeros(num_methods, num_channels); 
    
    for m = 1:num_methods
        current_Pxx = PSD_methods{m};
        
        for ch = 1:num_channels
            % Absolute power
            absolute_power = sum(current_Pxx(idx_band, ch));

            % Total power
            total_power = sum(current_Pxx(idx_total, ch));

            % Relative power
            Relative_Power_Matrix(m, ch) = absolute_power / total_power;
        end
    end
    
    % --- Determine Common Limits for plotting ---
    % Find the overall maximum relative power for consistent color scaling
    max_relative_power = max(Relative_Power_Matrix(:));
    common_limits = [0, max_relative_power];
    
    figure;
    sgtitle([band_name, ' Relative Power Topograms'], 'FontSize', 14);

    for m = 1:num_methods
        subplot(3, 3, m);
        draw_topogram(Relative_Power_Matrix(m, :)', common_limits); 
        
        title(method_names{m});
        
        % Add a colorbar only to the last plot (Method 7) for reference
        if m == num_methods
            colorbar('Location', 'EastOutside');
        end
    end
    
end
\end{lstlisting}


\subsection{Results}

\begin{figure}[H]
    \begin{subfigure}{0.45\textwidth}
        \centering
        \includegraphics[width=\linewidth]{img/ex1/LD}
    \end{subfigure}
    \hfill
     \begin{subfigure}{0.45\textwidth} 
        \centering
        \includegraphics[width=\linewidth]{img/ex1/delta}
    \end{subfigure}
\end{figure}

\begin{figure}[H]
    \begin{subfigure}{0.45\textwidth}
        \centering
        \includegraphics[width=\linewidth]{img/ex1/theta}
    \end{subfigure}
    \hfill
    \begin{subfigure}{0.45\textwidth} 
        \centering
        \includegraphics[width=\linewidth]{img/ex1/alpha}
    \end{subfigure}
\end{figure}

\begin{figure}[H]
    \begin{subfigure}{0.45\textwidth}
        \centering
        \includegraphics[width=\linewidth]{img/ex1/sigma}
    \end{subfigure}
    \hfill
    \begin{subfigure}{0.45\textwidth} 
        \centering
        \includegraphics[width=\linewidth]{img/ex1/beta}
    \end{subfigure}
\end{figure}

It can be observed that we have satisfied the requirement to use the saem limits 
[min, max] across all seven topograms in a signle figure, where min is 0 and max 
corresponds to the maximum value for all channesl and all methods in the figure.


\subsection{Discussion}
\noindent \subsubsection*{Feasibility}
For the dominant bands—\textbf{Low Delta, Delta, Theta, and Alpha}—all seven estimators 
(non-parametric Periodogram, Hamming, and Welch, along with the parametric Burg methods) 
produced \textbf{highly similar spatial distributions}. For instance, the Alpha band consistently 
showed a posterior-occipital maximum, while the \textbf{Delta band} (2-4 Hz) was also strongly 
localized posteriorly. This high level of agreement across various algorithms \textbf{confirms 
the feasibility and robustness} of the calculated relative power distributions for these 
major frequency components.

\noindent \subsubsection*{Can we observe the same tendency in all frequency bands during deeper sleep stages?}
When examining the "tendency" across all six frequency bands, it is clear that \textbf{the same topographic 
pattern is not observed}. The maximum relative power shifts significantly depending on the band: Low Delta, 
Theta, and Beta activity were concentrated primarily in the anterior \textbf{(frontal)} regions of the scalp, whereas, 
the Delta (2-4 Hz), Alpha, and Sigma activity localized to the \textbf{posterior (occipital/parietal) and central} areas.


\noindent \subsubsection*{Is there any band more affected than others by the estimator selection (methods)?}
The \textbf{Sigma band (12-15 Hz)} was the \textbf{most affected} because its low power (max ~ 0.06) makes the estimate volatile. 
Non-parametric methods like \textbf{Welch} create a \textbf{diffuse} pattern due to smoothing. In contrast, \textbf{parametric Burg methods} 
(AIC/MDL) offer \textbf{superior frequency resolution} and successfully reveal the \textbf{sharper, localized peaks} of the sleep spindles. 
This difference proves that estimator choice is critical for accurately localizing narrow-band, low-magnitude features.



