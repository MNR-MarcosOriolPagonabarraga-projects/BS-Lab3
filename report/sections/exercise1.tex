\section{Feasibility Assessment}
In this section, we will be working with S1.mat recording.
\subsubsection*{Data Loading}
In the "S1.mat" file there are 21 biological channels recorded during 30 seconds with a sampling frequency of 100 Hz. 
In the document it is mentionted that: 
\textit{It is usually used to consider EEG signals stationary during 5-secnods epochs. 
Thereofre, from here we can determine that there are 5 epochs of 5 seconds each.}


\subsection{a) Compare the PSD functions for the channels Fz, Cz and Pz obtained in all 7 spectral methods in a figure (consider only the epoch 1).}

\begin{itemize}
    \item Epoch 1: corresponds to the first 5 seconds.
    \item First, we compute the PSD for the last 19 EEG channels, for the first 5 seconds (epoch 1). 
    \item Then we selected from the third channel (this included) until the last one, in total 29 channels. This is because the first two correspond to the two ocular channels.
\end{itemize}


\begin{lstlisting}
% load data
load("data/EEG_signals-20251023/S1.mat")

fs = 100; % Sampling frequency (Hz)
N = 5*fs; % samples in one 5-seconds epoch

% Compute all PSD from the 19 EEG channels (ephoch 1)
% We select the first 5 seconds epoch and the last 19EEG channels:
eegi = signals(1:N,3:21); % From 1 to 500 samples (5 seconds)
\end{lstlisting}

\vspace{0.3cm}
Then, it is important to filter the signal. As frequency bands come from 0.5 Hz to 35 Hz, it is convenient to remove the very low frequency components associated 
mainly with the movement and other brain activities different to delta rhythms. By doing this it will permit to estimate better the PSD.
\vspace{0.3cm}

\begin{lstlisting}
% Apply high-pass filter
[b,a] = ellip(6,0.5,40,.4/(fs/2),'high');
eeg = filtfilt(b,a,eegi);
\end{lstlisting}

\vspace{0.3cm}
\noindent With the data filtered, we will proceed to compute the PSD using the 7 spectral methods, defined in ~\ref{sec:psd_methods}.
\vspace{0.3cm}

\begin{lstlisting}[caption={Code used for running the different Power Spectral Methods.}]
% Method 1 - Periodogram, considering a FFT with 1000 points:
[Pxx,f]=periodogram(eeg,[],1000,fs,'onesided');

% Method 2 - Periodogram using a Hamming window
[PxxH,f]=periodogram(eeg,hamming(500),1000,fs,'onesided');

% Method 3 - averaged Welch periodogram
number_segments = 4;
overlap = 0.5;
samples_segment = floor(length(eeg)/(number_segments-(number_segments-1)*overlap));
[PxxW,f] = pwelch(eeg,samples_segment,floor(samples_segment*overlap), 1000, fs, 'onesided'); 

% Method 4 - Burg method with low order of 2
[Parb2,f]= pburg(eeg,2,1000,fs,'onesided');

% Method 5 - Burg method with high order of 50
[Parb50,f]= pburg(eeg,50,1000,fs,'onesided');

% Method 6 and 7 - AIC and MDL, respectively
nch = size(eeg, 2);
nnaic = zeros(nch, 1);
nnmdl = zeros(nch, 1);

for n = 1:nch
 data1= iddata(eeg(1:N/2, n),[],fs);
 data2= iddata(eeg(N/2 + 1: N,n),[], fs);
 V1=arxstruc(data1,data2, (1:60));
 nnaic(n)=selstruc(V1,'aic');
 nnmdl(n)=selstruc(V1,'mdl');
end

% Method 6 (AIC)
naic = round(mean(nnaic));
[Parbaic,f] = pburg(eeg,naic,1000,fs,'onesided');

% Method 7 (MDL)
nmdl=round(mean(nnmdl));
[Parbmdl,f]= pburg(eeg,nmdl,1000,fs,'onesided');
\end{lstlisting}

\vspace{0.3cm}

We now select the channels indicated (Fz, Cz and Pz). Since the data we are working with ("eeg") has already 19 channels 
(the two first, corresponding to the ocular channels, have been removed previously, variable "eegi"), we can assign the 
following correspondences between channelsa and rows from "eegi":

\begin{itemize}
    \item Fz: 5th channel
    \item Cz: 10th channel
    \item Pz: 15th channel
\end{itemize}

\noindent \subsubsection*{Plotting code}
To perform the following code, we have used ChatGPT, as the code for plotting the PSD from the 7 methods is very similar, just changing the data used (Pxx, PxxH, PxxW, etc.) and the title with the corresponding method used.

\begin{lstlisting}[caption={Matlab code for plotting PSD.}]
methods = {'Periodogram','Periodogram + Hamming','Welch', ...
           'Burg(2)','Burg(50)', 'Burg(AIC)','Burg(MDL)'};

idxFz = 5;
idxCz = 10;
idxPz = 15;

nexttile;
plot(f, Pxx(:,idxFz), 'LineWidth',1); hold on;
plot(f, Pxx(:,idxCz), 'LineWidth',1);
plot(f, Pxx(:,idxPz), 'LineWidth',1); hold off;
title('Periodogram');
xlabel('Frequency (Hz)');
ylabel('Power Spectral Density (Power/Hz)');
xlim([0 50]);
grid on;
legend({'Fz','Cz','Pz'},'Location','northeast');
\end{lstlisting}

\noindent \subsubsection*{Plotting Result}

\begin{figure}[H]
    \centering
    \includegraphics[width=0.8\textwidth]{img/ex1/ps_methods}
    \caption{7 PSD methods applied to the epoch 1 of S1 data.}
\end{figure}


\subsection{b) Compare the topograms of all 7 spectral methods in a figure for each relative power (indicate the max value considered in each of both figures).}
The frequency bands and their corresponding limits are the ones indicated in ~\ref{sec:defined_bands}.
Next, we present the code used for calculating the PSD methods for the defined bands.

\begin{lstlisting}[caption={Matlab code used for running the different PSD for each frequency band.}]
% Define the 7 spectral methods and their names
PSD_methods = {Pxx, PxxH, PxxW, Parb2, Parb50, Parbaic, Parbmdl};
method_names = {'Periodogram', 'Hamming', 'Welch', 'Burg (p=2)', 'Burg (p=50)', 'Burg (AIC)', 'Burg (MDL)'};
num_methods = length(PSD_methods);
num_channels = size(eeg, 2);

% Define all 6 frequency bands and their limits
Band_Limits = {
    'Low Delta (0.5 - 2 Hz)', [0.5, 2.0];
    'Delta (2 - 4 Hz)',       [2.0, 4.0];
    'Theta (4 - 7.5 Hz)',     [4.0, 7.5];
    'Alpha (7.5 - 12 Hz)',    [7.5, 12.0];
    'Sigma (12 - 15 Hz)',     [12.0, 15.0];
    'Beta (15 - 35 Hz)',      [15.0, 35.0];
};

num_bands = size(Band_Limits, 1);

% Total frequency range for normalization
f_min_total = 0.5; % Hz
f_max_total = 30.0; % Hz 
idx_total = find(f >= f_min_total & f <= f_max_total); % f: frequency vector

for band_idx = 1:num_bands
    
    % --- Get Current Band Info ---
    band_name = Band_Limits{band_idx, 1};
    f_min_band = Band_Limits{band_idx, 2}(1);
    f_max_band = Band_Limits{band_idx, 2}(2);

    disp(['Processing band: ', band_name]);

    % Find frequency indices for the current band
    idx_band = find(f >= f_min_band & f <= f_max_band);
    
   % Initialize matrix to store relative power for all methods (Rows=Methods, Cols=Channels)
   % Relative power calculated as before
    Relative_Power_Matrix = zeros(num_methods, num_channels); 
    
    for m = 1:num_methods
        current_Pxx = PSD_methods{m};
        
        for ch = 1:num_channels
            % Absolute power
            absolute_power = sum(current_Pxx(idx_band, ch));

            % Total power
            total_power = sum(current_Pxx(idx_total, ch));

            % Relative power
            Relative_Power_Matrix(m, ch) = absolute_power / total_power;
        end
    end
    
    % --- Determine Common Limits for plotting ---
    % Find the overall maximum relative power for consistent color scaling
    max_relative_power = max(Relative_Power_Matrix(:));
    common_limits = [0, max_relative_power];
    
    figure;
    sgtitle([band_name, ' Relative Power Topograms'], 'FontSize', 14);

    for m = 1:num_methods
        subplot(3, 3, m);
        draw_topogram(Relative_Power_Matrix(m, :), common_limits); 
        
        title(method_names{m});
        
        % Add a colorbar only to the last plot (Method 7) for reference
        if m == num_methods
            colorbar('Location', 'EastOutside');
        end
    end
    
end
\end{lstlisting}


\subsection{Results}
Below we are presenting the different relative power 
topograms obtained for the different frequency bands.

\begin{figure}[H]
    \begin{subfigure}{0.45\textwidth}
        \centering
        \includegraphics[width=\linewidth]{img/ex1/LD}
    \end{subfigure}
    \hfill
     \begin{subfigure}{0.45\textwidth} 
        \centering
        \includegraphics[width=\linewidth]{img/ex1/delta}
    \end{subfigure}
    \label{fig:topogram_ex1}
\end{figure}

\begin{figure}[H]
    \begin{subfigure}{0.45\textwidth}
        \centering
        \includegraphics[width=\linewidth]{img/ex1/theta}
    \end{subfigure}
    \hfill
    \begin{subfigure}{0.45\textwidth} 
        \centering
        \includegraphics[width=\linewidth]{img/ex1/alpha}
    \end{subfigure}
\end{figure}

\begin{figure}[H]
    \begin{subfigure}{0.45\textwidth}
        \centering
        \includegraphics[width=\linewidth]{img/ex1/sigma}
    \end{subfigure}
    \hfill
    \begin{subfigure}{0.45\textwidth} 
        \centering
        \includegraphics[width=\linewidth]{img/ex1/beta}
    \end{subfigure}
\end{figure}

It can be observed that we have satisfied the requirement to use the saem limits 
[min, max] across all seven topograms in a signle figure, where min is 0 and max 
corresponds to the maximum value for all channels and all methods in the figure.


\subsection{Discussion}
From Figure 1, we can see that depending on the estimator used, the PSD functions
obtained are different.
\begin{itemize}
    \item When we use the non-parametric methods (periodogram and Welch methods) it
can be seen that, for the periodogram method, the PSD functions obtained are
quite noisy (have several peaks), whereas when using the Welch’s method, the
PSD functions are smoother, which means that in this case there it is not capturing
that much details.
    \item  When using the parametric methods, we can see that the PSD obtained are much
smoother in comparison with the ones obtained with the non-parametric
methods. However, when using the Burg method with order 50, the spectrum
obtained shows sharp peaks, therefore, it is trying to include more details, but
consequently, it is introducing more noise.
\end{itemize}

Regarding the topograms, these show the spatial distribution of relative power
across all six EEG frequency bands, and how that distribution is affected by the seven
methos we have used. By comparing the topograms generated by the seven methods for
the same frequency band, we want to assess the feasibility of the results. Therefore,
looking at the topograms, we see that most of the methods produce highly similar
topographic patterns. In addition, we also see that the Burg method with order 2 can be
considered infeasible since for all the frequency bands, this method produces a spatial
distribution different from the other methods. In conclusion, six out of the seven methods
used agree on the power distribution (have similar results), therefore, we can say that the
results obtained are feasible.
